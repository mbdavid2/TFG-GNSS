% -*-latex-*-
% 
% For questions, comments, concerns or complaints:
% thesis@mit.edu
% 
%
% $Log: cover.tex,v $
% Revision 1.8  2008/05/13 15:02:15  jdreed
% Degree month is June, not May.  Added note about prevdegrees.
% Arthur Smith's title updated
%
% Revision 1.7  2001/02/08 18:53:16  boojum
% changed some \newpages to \cleardoublepages
%
% Revision 1.6  1999/10/21 14:49:31  boojum
% changed comment referring to documentstyle
%
% Revision 1.5  1999/10/21 14:39:04  boojum
% *** empty log message ***
%
% Revision 1.4  1997/04/18  17:54:10  othomas
% added page numbers on abstract and cover, and made 1 abstract
% page the default rather than 2.  (anne hunter tells me this
% is the new institute standard.)
%
% Revision 1.4  1997/04/18  17:54:10  othomas
% added page numbers on abstract and cover, and made 1 abstract
% page the default rather than 2.  (anne hunter tells me this
% is the new institute standard.)
%
% Revision 1.3  93/05/17  17:06:29  starflt
% Added acknowledgements section (suggested by tompalka)
% 
% Revision 1.2  92/04/22  13:13:13  epeisach
% Fixes for 1991 course 6 requirements
% Phrase "and to grant others the right to do so" has been added to 
% permission clause
% Second copy of abstract is not counted as separate pages so numbering works
% out
% 
% Revision 1.1  92/04/22  13:08:20  epeisach

% NOTE:
% These templates make an effort to conform to the MIT Thesis specifications,
% however the specifications can change.  We recommend that you verify the
% layout of your title page with your thesis advisor and/or the MIT 
% Libraries before printing your final copy.
\title{An Optimizing Compiler for Low-Level Floating Point Operations}

\author{Lucien William Van Elsen}
% If you wish to list your previous degrees on the cover page, use the 
% previous degrees command:
%       \prevdegrees{A.A., Harvard University (1985)}
% You can use the \\ command to list multiple previous degrees
%       \prevdegrees{B.S., University of California (1978) \\
%                    S.M., Massachusetts Institute of Technology (1981)}
\department{Department of Electrical Engineering and Computer Science}

% If the thesis is for two degrees simultaneously, list them both
% separated by \and like this:
% \degree{Doctor of Philosophy \and Master of Science}
\degree{Bachelor of Science in Computer Science and Engineering}

% As of the 2007-08 academic year, valid degree months are September, 
% February, or June.  The default is June.
\degreemonth{June}
\degreeyear{1990}
\thesisdate{May 18, 1990}

%% By default, the thesis will be copyrighted to MIT.  If you need to copyright
%% the thesis to yourself, just specify the `vi' documentclass option.  If for
%% some reason you want to exactly specify the copyright notice text, you can
%% use the \copyrightnoticetext command.  
%\copyrightnoticetext{\copyright IBM, 1990.  Do not open till Xmas.}

% If there is more than one supervisor, use the \supervisor command
% once for each.
\supervisor{William J. Dally}{Associate Professor}

% This is the department committee chairman, not the thesis committee
% chairman.  You should replace this with your Department's Committee
% Chairman.
\chairman{Arthur C. Smith}{Chairman, Department Committee on Graduate Theses}

% Make the titlepage based on the above information.  If you need
% something special and can't use the standard form, you can specify
% the exact text of the titlepage yourself.  Put it in a titlepage
% environment and leave blank lines where you want vertical space.
% The spaces will be adjusted to fill the entire page.  The dotted
% lines for the signatures are made with the \signature command.
\maketitle

% The abstractpage environment sets up everything on the page except
% the text itself.  The title and other header material are put at the
% top of the page, and the supervisors are listed at the bottom.  A
% new page is begun both before and after.  Of course, an abstract may
% be more than one page itself.  If you need more control over the
% format of the page, you can use the abstract environment, which puts
% the word "Abstract" at the beginning and single spaces its text.

%% You can either \input (*not* \include) your abstract file, or you can put
%% the text of the abstract directly between the \begin{abstractpage} and
%% \end{abstractpage} commands.

% First copy: start a new page, and save the page number.
\cleardoublepage
% Uncomment the next line if you do NOT want a page number on your
% abstract and acknowledgments pages.
% \pagestyle{empty}
\setcounter{savepage}{\thepage}
\begin{abstractpage}
\thispagestyle{empty}
% tio(X,Z), padre(X,Y) tralari que te di dilo bien
\begin{center}
    \Large
    \textbf{Abstract}
\end{center}
% Maybe (Abstract) = Nothing \begin{abstract}
Stellar flares are sudden electromagnetic emissions on a star's surface  large amounts of energy. These flares are detected by telescopes such as Swift or Fermi by performing radiation observations from low Earth orbit. However, the radiation also has an effect on Earth’s ionosphere electron content. Another approach for detecting these events is possible: the aforementioned electron content variation can be processed using data from Global Navigation Satellite Systems (GNSS) such as GPS to study the flares.

The \textit{Blind GNSS Search of Extraterrestrial EUV Sources} (BGSEES) algorithm for detecting solar flares without knowing the location of the source, that is, the position of the Sun relative to Earth, is presented, as well as a study on the feasibility of the detection of such events for the challenging scenario of far-away stars, aiming to find an alternative detection method to that of the telescopes and using free, open access data.

\textbf{Keywords}: Solar flares, Stellar flares, GNSS, GPS, IGS, GRB
% \end{abstract}

\clearpage
\thispagestyle{empty}
\begin{center}
	\Large
	\textbf{Resumen}
\end{center}
% \begin{abstract}
Las fulguraciones estelares son incrementos repentinos de radiaciones electromagnéticas en la superficie de una estrella que contienen gran cantidad de energía. Son detectadas por telescopios tales como Swift o Fermi realizando observaciones desde satélites artificiales en órbita baja terrestre. Esta radiación también afecta al contenido total de electrones de la ionosfera terrestre. Esto permite una alternativa para detectar las fulguraciones: detectar el incremento repentino y según un cierto patrón espacial característico del mencionado contenido de electrones, que puede ser procesado usando las medidas transionosféricas en doble frecuencia de los Sistemas de Navegación Global por Satélite (Global Navigation Satellite Systems, GNSS), como por ejemplo GPS.

Se propone el algoritmo \textit{Blind GNSS Search of Extraterrestrial EUV Sources} (BGSEES) para detectar fulguraciones solares sin conocimiento previo de la posición de la fuente, es decir, la posición del Sol relativa a la de la Tierra en la caracterización del algoritmo ante fulguraciones solares conocidas, así como un estudio sobre la viabilidad de la detección de fulguraciones que tienen lugar en estrellas lejanas. El objetivo final es el de confirmar la validez de un nuevo método de detección alternativo al de los telescopios que usa datos de libre acceso y omnidireccional.

\textbf{Keywords}: Solar flares, Stellar flares, GNSS, GPS, IGS, GRB

\clearpage
\thispagestyle{empty}
\begin{center}
	\Large
	\textbf{Resum}
\end{center}
% \begin{abstract}
Les fulguracions estel·lars són increments de radiacions electromagnètiques en la superfície d'una estrella que contenen una gran quantitat d'energia. Són detectades per telescopis com Swift o Fermi realitzant observacions des de satèl·lits artificials en òrbita baixa terrestre. La radiació també afecta el contingut total d'electrons de la ionosfera terrestre. Això permet una alternativa per detectar les mencionades fulguracions: detectar l'increment i seguint un patró espacial característic del contingut d'electrons, que pot ser processat utilitzant les mesures transionosfèriques en doble freqüència dels Sistemes de Navegació Global per Satèl·lit (Global Navigation Satellite Systems, GNSS), com per exemple GPS.

Es proposa l'algoritme \textit{Blind GNSS Search of Extraterrestrial EUV Sources} (BGSEES) per detectar fulguracions solars sense cap coneixement previ de la posició de la font, és a dir, la posició del Sol relativa a la de la Terra en la caracterització de l'algoritme davant de fulguracions solars conegudes, així com un estudi sobre la viabilitat de la detecció de fulguracions que tenen lloc a estrelles llunyanes. L'objectiu final és el de confirmar la validesa d'un nou mètode de detecció alternatiu al dels telescopis omnidireccional que fagi servir dades de lliure accés.

\textbf{Keywords}: Solar flares, Stellar flares, GNSS, GPS, IGS, GRB
\end{abstractpage}

% Additional copy: start a new page, and reset the page number.  This way,
% the second copy of the abstract is not counted as separate pages.
% Uncomment the next 6 lines if you need two copies of the abstract
% page.
% \setcounter{page}{\thesavepage}
% \begin{abstractpage}
% \thispagestyle{empty}
% tio(X,Z), padre(X,Y) tralari que te di dilo bien
\begin{center}
    \Large
    \textbf{Abstract}
\end{center}
% Maybe (Abstract) = Nothing \begin{abstract}
Stellar flares are sudden electromagnetic emissions on a star's surface  large amounts of energy. These flares are detected by telescopes such as Swift or Fermi by performing radiation observations from low Earth orbit. However, the radiation also has an effect on Earth’s ionosphere electron content. Another approach for detecting these events is possible: the aforementioned electron content variation can be processed using data from Global Navigation Satellite Systems (GNSS) such as GPS to study the flares.

The \textit{Blind GNSS Search of Extraterrestrial EUV Sources} (BGSEES) algorithm for detecting solar flares without knowing the location of the source, that is, the position of the Sun relative to Earth, is presented, as well as a study on the feasibility of the detection of such events for the challenging scenario of far-away stars, aiming to find an alternative detection method to that of the telescopes and using free, open access data.

\textbf{Keywords}: Solar flares, Stellar flares, GNSS, GPS, IGS, GRB
% \end{abstract}

\clearpage
\thispagestyle{empty}
\begin{center}
	\Large
	\textbf{Resumen}
\end{center}
% \begin{abstract}
Las fulguraciones estelares son incrementos repentinos de radiaciones electromagnéticas en la superficie de una estrella que contienen gran cantidad de energía. Son detectadas por telescopios tales como Swift o Fermi realizando observaciones desde satélites artificiales en órbita baja terrestre. Esta radiación también afecta al contenido total de electrones de la ionosfera terrestre. Esto permite una alternativa para detectar las fulguraciones: detectar el incremento repentino y según un cierto patrón espacial característico del mencionado contenido de electrones, que puede ser procesado usando las medidas transionosféricas en doble frecuencia de los Sistemas de Navegación Global por Satélite (Global Navigation Satellite Systems, GNSS), como por ejemplo GPS.

Se propone el algoritmo \textit{Blind GNSS Search of Extraterrestrial EUV Sources} (BGSEES) para detectar fulguraciones solares sin conocimiento previo de la posición de la fuente, es decir, la posición del Sol relativa a la de la Tierra en la caracterización del algoritmo ante fulguraciones solares conocidas, así como un estudio sobre la viabilidad de la detección de fulguraciones que tienen lugar en estrellas lejanas. El objetivo final es el de confirmar la validez de un nuevo método de detección alternativo al de los telescopios que usa datos de libre acceso y omnidireccional.

\textbf{Keywords}: Solar flares, Stellar flares, GNSS, GPS, IGS, GRB

\clearpage
\thispagestyle{empty}
\begin{center}
	\Large
	\textbf{Resum}
\end{center}
% \begin{abstract}
Les fulguracions estel·lars són increments de radiacions electromagnètiques en la superfície d'una estrella que contenen una gran quantitat d'energia. Són detectades per telescopis com Swift o Fermi realitzant observacions des de satèl·lits artificials en òrbita baixa terrestre. La radiació també afecta el contingut total d'electrons de la ionosfera terrestre. Això permet una alternativa per detectar les mencionades fulguracions: detectar l'increment i seguint un patró espacial característic del contingut d'electrons, que pot ser processat utilitzant les mesures transionosfèriques en doble freqüència dels Sistemes de Navegació Global per Satèl·lit (Global Navigation Satellite Systems, GNSS), com per exemple GPS.

Es proposa l'algoritme \textit{Blind GNSS Search of Extraterrestrial EUV Sources} (BGSEES) per detectar fulguracions solars sense cap coneixement previ de la posició de la font, és a dir, la posició del Sol relativa a la de la Terra en la caracterització de l'algoritme davant de fulguracions solars conegudes, així com un estudi sobre la viabilitat de la detecció de fulguracions que tenen lloc a estrelles llunyanes. L'objectiu final és el de confirmar la validesa d'un nou mètode de detecció alternatiu al dels telescopis omnidireccional que fagi servir dades de lliure accés.

\textbf{Keywords}: Solar flares, Stellar flares, GNSS, GPS, IGS, GRB
% \end{abstractpage}

\cleardoublepage



%%%%%%%%%%%%%%%%%%%%%%%%%%%%%%%%%%%%%%%%%%%%%%%%%%%%%%%%%%%%%%%%%%%%%%
% -*-latex-*-
