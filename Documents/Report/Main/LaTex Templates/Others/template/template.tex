\documentclass[12pt]{article}

%%% Packages %%%
\usepackage[utf8]{inputenc}
\usepackage{titlesec}

\titleformat{\chapter}{\normalfont\huge}{\thechapter.}{20pt}{\huge\it}

\title{Deliverable 1: Context and scope of the project}
\author{David Moreno Borr\`as}
\date{February 2019}

%%%%%%%%%%%%%%%%%%%%%%%%%%%%%%%%%%%
%%%%%%% Start of document %%%%%%%%%
%%%%%%%%%%%%%%%%%%%%%%%%%%%%%%%%%%%

\begin{document}

\maketitle

\newpage
\tableofcontents

\newpage
\pagenumbering{arabic}
\setcounter{page}{1}


%%%%%%%%%%%%%%%%%%%%%%%%%%%%%%
%%%%%%% Introduction %%%%%%%%%
%%%%%%%%%%%%%%%%%%%%%%%%%%%%%%
\section{Introduction}

Solar flares are sudden electromagnetic emissions on the Sun’s surface that release large amounts of magnetic energy. These flares emit radiation that has an effect on Earth’s ionosphere electron content and therefore the many satellites orbiting it. [referencia al pdf de manuel]\\

Several NASA missions that aim to detect these as well as flares from far-away stars exist, like the Swift and Fermi missions, these satellites, though, perform this by using their instruments to study the gamma-ray, x-ray and ultraviolet radiation bands. [referencias]\\

The aforementioned ionosphere electron content variation, however, makes it possible to detect these flares with a more indirect approach: using data from the satellites belonging to global positioning systems.\\

As the sudden increase of electron content in the ionosphere has an effect on the signals these satellites receive and send, this data can be used to detect flares by using the appropriate algorithms. Parameters such as the angle between the Sun and the zenith of the earth, or the Total Electron Content (TEC) in the air have to be taken into consideration for them to work. \\

This is already feasible with flares that have the sun as a source, so our goal is to study if it is possible to expand on that by detecting flares from far-away stars and developing the required algorithms.\\

Therefore, the main objective of our project is to study the feasibility of the detection of these stellar flares using Global Navigation Satellite System (GNSS) measurements without knowing the location of the source. If this is possible, it could be extended to real-time detection.\\

The aim of this document is to give a detailed description of the project, its scope and context.

%%%%%%%%%%%%%%%%%%%%%%%%%%%%%%%%%%%%%
%%%%%%% Scope of the project %%%%%%%%
%%%%%%%%%%%%%%%%%%%%%%%%%%%%%%%%%%%%%
\newpage
\section{Scope of the project}

\subsection{Dump and Dumps}

\subsection{Dump and Dumps}










%%%%%%%%%%%%%%%%%%%%%%%%%%%%
%%%%%%% References %%%%%%%%%
%%%%%%%%%%%%%%%%%%%%%%%%%%%%

\newpage
\begin{thebibliography}{999}

\bibitem{beyer}
  Sylvia Beyer,
  \emph{“Why are women underrepresented in Computer Science? Gender
differences in stereotypes, self-efficacy, values, and interests and
predictors of future CS course-taking and grades”}.
  Department of Psychology, University of Wisconsin-Parkside.
  2014.

\bibitem{varma}
  R. Varma,
  \emph{“Computing self-efficacy among women in India”}.
  Women and Minorities in Science and Engineering, vol. 16,
  2010.
  
\bibitem{varma2}
  R. Varma,
  \emph{“Why I Chose Computer Science? Women in India ”}.
  University of Mexico,
  2009.
  
\bibitem{AISHE}
  AISHE,
  \emph{Report by ALL INDIA SURVEY ON
HIGHER EDUCATION
2017-18}
  

  

\end{thebibliography}

\end{document}