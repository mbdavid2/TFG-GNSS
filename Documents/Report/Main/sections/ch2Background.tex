\chapter{Background}

As the project has a large background in physics and astronomy, some of the relevant topics that are going to be studied are introduced: Global Navigation Satellite Systems, the Ionosphere, Stellar Flares and Gamma-Ray Bursts.

\subsection{Global Navigation Satellite Systems}

During the project both GNSS and GPS will be mentioned. GNSS stands for Global Navigation Satellite Systems and GPS for Global Positioning System 

In 1998 the International GNSS Service (IGS) was created as a collaboration of several members of the scientific community: Center for Orbit Determination in Europe (CODE), (European Space Agency) ESA, Jet Propulsion Laboratory (JPL) and Polytechnic University of Catalonia (UPC). This voluntary federation has made available open access GNSS data since its creation. 

Its data is provided by more than 300 GPS receivers around the globe and is processed by the previous institutions which compute the global distribution of the TEC.

de donde sacamos los datos

Referencias en este apartado: {The IGS VTEC maps: a reliable source of ionospheric information since 1998}

explicado en pagina 5 del libro este Reference: GNSS – Global Navigation Satellite Systems: GPS, GLONASS, Galileo, and more

\subsection{Ionosphere}

The ionosphere is a layer of the Earth’s atmosphere that lies 75-1000km above its surface. (http://solar-center.stanford.edu/SID/activities/ionosphere.html) 

Due to the free electrons and ionized molecules it has, it is capable of affecting radio wave propagation, thus having an effect on Global Navigation Satellite Systems (GNSS) technology, this phenomena allows these satellites to be used as a global scanner for the ionosphere. 

The main physical quantity used for describing the electron content of the ionosphere is the Total Electron Content (TEC), the TEC is the total number of electrons between two points (r1,r2) along a cylinder of base 1m2. 
Slant TEC (STEC), in particular can defined as the TEC in which r1 and r2 are a satellite and a receiver’s positions.

 (aqui citar a GPS as a solar observational instrument: Real-time estimation of EUV photons flux rate during strong, medium, and weak solar flares)

The unique properties the ionosphere has enable us to use the data provided by the GNSS technology to study a powerful phenomena that occurs in many stars across the universe: stellar flares.

Referencias en este apartado: 1{International reference ionosphere 1990}, 2{The ionosphere: effects, GPS modeling and the benefits for space geodetic techniques}

\subsection{Stellar flares}

Flares from stars, in particular those that have the Sun as a source, more noticeable due to its proximity, are sudden flashes of brightness in the surface of stars which release large amounts of energy across the whole electromagnetic spectrum pie de pagina  (X-rays and Extreme Ultraviolet (EUV)) rays.
Flares that have the Sun as a source can increase the electron content of the ionosphere and have an effect on waves passing through it, affecting satellite communications and causing a delay. This phenomena is the key element of the project, as it enables us to study these events.

Satellites can also be harmed by this effects: the flares heat up the outer atmosphere, which in turn increases the drag on these satellites reducing their lifetime in orbit.
{GPS as a solar observational instrument: Real-time estimation of EUV photons flux rate during strong, medium, and weak solar flares se explica el efecto del sol sobre ionosfera}

The previous phenomena applies to flares originating in the Sun, whether a flare that has a star from outside the Solar System as a source has an effect on the Earth’s atmosphere or not is one of the topic that is going to be studied in this project.

\subsection{Gamma-Ray Bursts}

Throughout the project, in particular when studying the feasibility of stellar flares detection, another type of event will mentioned and studied as well: Gamma-Ray Bursts (GRBs):

GRBs are highly energetic explosions that occur in distant galaxies, releasing large amounts of radiation, in particular Gamma rays, hence the name of the event. These bursts, despite being millions of light years away from Earth are so powerful they might still have an impact on the ionosphere, like the aforementioned stellar flares. 
The main difference with flares originating from stars is that GRBs are thought to be originated from the death of massive stars, that is, supernovas.

This event, despite not being a stellar flare, the phenomena we aim to detect, is going to be studied in the following sections to test the currently working algorithms mainly for two reasons: it has been studied and cataloged by telescopes such as the Fermi Observatory and there’s is available information that we can use for our study, and the large amounts of energy they emit it make GRBs a more feasible target to detect.

{Fermi Observations of High-Energy Gamma-Ray Emission from GRB 080916C}




