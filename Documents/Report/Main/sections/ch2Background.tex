\chapter{Background}
As the project has a large background in physics and astronomy, some of the relevant topics that are going to be studied are introduced: Global Navigation Satellite Systems, the Ionosphere, Stellar Flares and Gamma-Ray Bursts.

\section{Global Navigation Satellite Systems}

\paragraph{GNSS and GPS}

These two terms may lead to confusion as Global Navigation Satellite Systems (GNSS) is the generic term for all satellite navigation systems. The Global Positioning System (GPS), in particular, is the United States' GNSS system, the world's most used one. Other systems, for example, are the European Galileo or Russian GLONASS \cite{hegarty2008evolution}.

Global Navigation Satellite Systems use satellites to determine the position of a given object or device in terms of latitude, longitude and height. 

\paragraph{Positioning}

In short, GNSS works as follows: out of all the GNSS satellites orbiting Earth, at least four of them are constantly visible from a specific point and transmitting information at a certain frequency. When a device receives a signal from one of them, the distance to the satellite can be calculated by means of the time required to reach it and the speed of light. As many variables might affect the speed of light such as the medium through which it is propagating, this estimation of the distance is called \textbf{pseudo-range}. 

Thus, the location of the receiver can be estimated using a technique called \textbf{trilateration}. Having three spheres around each of the satellites with the pseudo-range as their radius, the intersection of these spheres yields the location of the receiver \cite{hofmann2007gnss}. 

\paragraph{Ionospheric Piercing Points}

Ionospheric Piercing Points (IPP) are going to be very relevant throughout the development of the project is dsadasdasd

\paragraph{The International GNSS Service}

In 1998 the International GNSS Service (IGS) was created as a collaboration of several members of the scientific community: Center for Orbit Determination in Europe (CODE), (European Space Agency) ESA, Jet Propulsion Laboratory (JPL) and Polytechnic University of Catalonia (UPC). This voluntary federation has made available open access GNSS data since its creation   \cite{igswebsite} \cite{dow2009international}. 

Its data is provided by more than 300 GPS receivers around the globe and is processed by the previous institutions which compute the global distribution of the Total Electron Content (TEC) \cite{hernandez2009igs}.

GNSS is a key component to this project because, as mentioned before, many variables can affect the speed of light and therefore the time it takes for the transmitter's signal to be intercepted by the receiver. One of these variables is the electron content of the layer of the atmosphere where GNSS satellites operate: the ionosphere.

\section{Ionosphere}

The ionosphere is a layer of the Earth’s atmosphere that lies 75-1000km above the surface of the planet \cite{ionospherestandford}. 

High energy from Extreme UltraViolet (EUV) and X-ray radiation can cause its atoms to be ionized and create a layer of electrons \cite{noaa2ionosphere}. Due to these free electrons and ionized molecules, it is capable of affecting radio wave propagation, thus having an effect on Global Navigation Satellite Systems (GNSS) technology, this phenomena allows these satellites to be used as a global scanner for the ionosphere \cite{hernandez2011ionosphere}. 

The main physical quantity used for describing the electron content of the ionosphere is the \textbf{Total Electron Content (TEC)}, the TEC is the total number of electrons between two points $(r1,r2)$ along a cylinder of base $1m^2$. 
Slant TEC (STEC), in particular, can be defined as the TEC in which $r1$ and $r2$ are a satellite and a receiver’s positions \cite{singh2015gps}. 

The unique properties the ionosphere has enable us to use the data provided by the GNSS technology to study a powerful phenomena that occurs in many stars across the universe: stellar flares.

\section{Stellar flares}

Flares from stars, in particular those that have the Sun as a source, more noticeable due to its proximity, are sudden flashes of brightness in the surface of stars which release large amounts of energy across the whole electromagnetic spectrum\footnote{X-rays and Extreme Ultraviolet (EUV) radiation}.
Flares that have the Sun as a source can increase the electron content of the ionosphere and have an effect on waves passing through it, affecting satellite communications and causing a delay. This phenomena is the key element of the project, as it enables us to study these events.

Satellites can also be harmed by this effects: the flares heat up the outer atmosphere, which in turn increases the drag on these satellites reducing their lifetime in orbit.
{GPS as a solar observational instrument: Real-time estimation of EUV photons flux rate during strong, medium, and weak solar flares se explica el efecto del sol sobre ionosfera}

The previous phenomena applies to flares originating in the Sun, whether a flare that has a star from outside the Solar System as a source has an effect on the Earth’s atmosphere or not is one of the topic that is going to be studied in this project.

SOHO and GOES are space probes used for obtaining solar flare information \cite{hernandez2012gnss}

\section{Gamma-Ray Bursts}

Throughout the project, in particular when studying the feasibility of stellar flares detection, another type of event will be mentioned and studied as well: Gamma-Ray Bursts (GRBs):

GRBs are highly energetic explosions that occur in distant galaxies, releasing large amounts of radiation, in particular Gamma rays, hence the name of the event. These bursts, despite being millions of light years away from Earth are so powerful they might still have an impact on the ionosphere, like the aforementioned stellar flares. 
The main difference with flares originating from stars is that GRBs are thought to be originated from the death of massive stars, that is, supernovas.

This event, despite not being a stellar flare, the phenomena we aim to detect, is going to be studied in the following sections to test the currently working algorithms mainly for two reasons: it has been studied and cataloged by telescopes such as the Fermi Observatory and there’s is available information that we can use for our study, and the large amounts of energy they emit it make GRBs a more feasible target to detect.

{Fermi Observations of High-Energy Gamma-Ray Emission from GRB 080916C}




