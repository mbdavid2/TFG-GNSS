\thispagestyle{empty}
\begin{center}
    \Large
    \textbf{Abstract}
\end{center}
% \begin{abstract}
Stellar flares are sudden electromagnetic emissions on a star's surface  large amounts of energy. These flares are detected by telescopes such as Swift or Fermi by performing radiation observations from low Earth orbit. However, the radiation also has an effect on Earth’s ionosphere electron content. Another approach for detecting these events is possible: the aforementioned electron content variation can be processed using data from Global Navigation Satellite Systems (GNSS) such as GPS to study the flares.

The \textit{Blind GNSS Search of Extraterrestrial EUV Sources} (BGSEES) algorithm for detecting solar flares without knowing the location of the source, that is, the position of the Sun relative to Earth, is presented, as well as a study on the feasibility of the detection of such events for the challenging scenario of far-away stars, aiming to find an alternative detection method to that of the telescopes and using free, open access data.

\textbf{Keywords}: Solar flares, Stellar flares, GNSS, GPS, IGS, GRB
% \end{abstract}

\clearpage
\thispagestyle{empty}
\begin{center}
	\Large
	\textbf{Resumen}
\end{center}
% \begin{abstract}
Las fulguraciones estelares son incrementos repentinos de radiaciones electromagnéticas en la superficie de una estrella que contienen gran cantidad de energía. Son detectadas por telescopios tales como Swift o Fermi realizando observaciones desde satélites artificiales en órbita baja terrestre. Esta radiación también afecta al contenido total de electrones de la ionosfera terrestre. Esto permite una alternativa para detectar las fulguraciones: detectar el incremento repentino y según un cierto patrón espacial característico del mencionado contenido de electrones, que puede ser procesado usando las medidas transionosféricas en doble frecuencia de los Sistemas de Navegación Global por Satélite (Global Navigation Satellite Systems, GNSS), como por ejemplo GPS.

Se propone el algoritmo \textit{Blind GNSS Search of Extraterrestrial EUV Sources} (BGSEES) para detectar fulguraciones solares sin conocimiento previo de la posición de la fuente, es decir, la posición del Sol relativa a la de la Tierra en la caracterización del algoritmo ante fulguraciones solares conocidas, así como un estudio sobre la viabilidad de la detección de fulguraciones que tienen lugar en estrellas lejanas. El objetivo final es el de confirmar la validez de un nuevo método de detección alternativo al de los telescopios que usa datos de libre acceso y omnidireccional.

\textbf{Keywords}: Solar flares, Stellar flares, GNSS, GPS, IGS, GRB

\clearpage
\thispagestyle{empty}
\begin{center}
	\Large
	\textbf{Resum}
\end{center}
% \begin{abstract}
Les fulguracions estel·lars són increments de radiacions electromagnètiques en la superfície d'una estrella que contenen una gran quantitat d'energia. Són detectades per telescopis com Swift o Fermi realitzant observacions des de satèl·lits artificials en òrbita baixa terrestre. La radiació també afecta el contingut total d'electrons de la ionosfera terrestre. Això permet una alternativa per detectar les mencionades fulguracions: detectar l'increment i seguint un patró espacial característic del contingut d'electrons, que pot ser processat utilitzant les mesures transionosfèriques en doble freqüència dels Sistemes de Navegació Global per Satèl·lit (Global Navigation Satellite Systems, GNSS), com per exemple GPS.

Es proposa l'algoritme \textit{Blind GNSS Search of Extraterrestrial EUV Sources} (BGSEES) per detectar fulguracions solars sense cap coneixement previ de la posició de la font, és a dir, la posició del Sol relativa a la de la Terra en la caracterització de l'algoritme davant de fulguracions solars conegudes, així com un estudi sobre la viabilitat de la detecció de fulguracions que tenen lloc a estrelles llunyanes. L'objectiu final és el de confirmar la validesa d'un nou mètode de detecció alternatiu al dels telescopis omnidireccional que fagi servir dades de lliure accés.

\textbf{Keywords}: Solar flares, Stellar flares, GNSS, GPS, IGS, GRB