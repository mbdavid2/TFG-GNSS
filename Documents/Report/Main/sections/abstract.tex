\thispagestyle{empty}
\begin{abstract}
Solar flares are sudden electromagnetic emissions on the Sun’s surface that release large amounts of magnetic energy. These flares emit radiation that has an effect on Earth’s ionosphere electron content and are detected by telescopes such as Swift or Fermi by performing gamma-ray observations from low Earth orbit. (Swift y fermi no son para solar flares son para GRB en general!!, quizas aqui deberia ir alguno que se centre en el sol especificamente)

Another approach for detecting these events is possible: the ionosphere electron content variation can be studied using data from the Global Navigation Satellite System (GNSS) (such as GPS) 

Algorithms for detecting solar flares without knowing the location of the source, that is, the position of the Sun relative to Earth, as well as a study on the feasibility of the detection of such events for the challenging scenario of far-away stars, are presented, aiming to find an alternative detection method to that of the telescopes and using free, open-source data.

Keywords: Solar flares, Stellar flares, GNSS, GPS, IGS, GRB
\end{abstract}

