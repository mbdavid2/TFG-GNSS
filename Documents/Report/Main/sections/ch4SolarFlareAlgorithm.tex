\chapter{Solar flare detection}

Before starting to write the main solar flare detection algorithm a first program was developed in which the location of the sun was known and we knew when the solar flare had taken place.

To test this, the data used was that of the called "Hallowen Storm, in whi \\


Halloween storm, poner ejemplos, citar al texto, porque esta tormenta
Fichero ti, sampling 30s
formato
explicar que vamos a hacer – IPPs, hill climbing y buscar si una vez en un pico, hay relacion lineal entre VTEC I coseno. En lugar de eso, subir mirando

primer paso, ver datos de un IPP
Awk prepocesado de datos
fortran
gnu plot para empezar a ver






Try algorithm knowing the location of the source, sun. Show first distribution of VTEC thtough the day (vill), then for the specific moment in time

After that, talk about proposals for the main algorithm






\section{Pseudocode}

\begin{algorithm}
	\caption{My algorithm}\label{euclid}
	\begin{algorithmic}[1]
		\Procedure{MyProcedure}{}
		\State $\textit{stringlen} \gets \text{length of }\textit{string}$
		\State $i \gets \textit{patlen}$
		\BState \emph{top}:
		\If {$i > \textit{stringlen}$} \Return false
		\EndIf
		\State $j \gets \textit{patlen}$
		\BState \emph{loop}:
		\If {$\textit{string}(i) = \textit{path}(j)$}
		\State $j \gets j-1$.
		\State $i \gets i-1$.
		\State \textbf{goto} \emph{loop}.
		\State \textbf{close};
		\EndIf
		\State $i \gets i+\max(\textit{delta}_1(\textit{string}(i)),\textit{delta}_2(j))$.
		\State \textbf{goto} \emph{top}.
		\EndProcedure
	\end{algorithmic}
\end{algorithm}

\begin{lstlisting}[language=Python, caption=Python example]
import numpy as np

def incmatrix(genl1,genl2):
	m = len(genl1)
	n = len(genl2)
	M = None #to become the incidence matrix
	VT = np.zeros((n*m,1), int)  #dummy variable
	
	#compute the bitwise xor matrix
	M1 = bitxormatrix(genl1)
	M2 = np.triu(bitxormatrix(genl2),1) 
	
	for i in range(m-1):
	for j in range(i+1, m):
	[r,c] = np.where(M2 == M1[i,j])
	for k in range(len(r)):
	VT[(i)*n + r[k]] = 1;
	VT[(i)*n + c[k]] = 1;
	VT[(j)*n + r[k]] = 1;
	VT[(j)*n + c[k]] = 1;
	
	if M is None:
	M = np.copy(VT)
	else:
	M = np.concatenate((M, VT), 1)
	
	VT = np.zeros((n*m,1), int)
	
	return M
\end{lstlisting}

