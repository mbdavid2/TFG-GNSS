\chapter{Solar flare detection}

Before starting to write the main solar flare detection algorithm a first program was developed that would target a powerful solar flare for which we know the time of the event and therefore, the location of the Sun.

Although the aim of the main algorithm is detecting the solar flare without taking into consideration the position of the Sun, this first approach was done to understand how the inner part of the algorithm works: studying the correlation between the cosine of the solar-zenith angle and the VTEC content.

This sections also provides an introduction to the formatting and use of the Global Navigation Satellite Systems data (GPS in this case) and how the main parameters are computed.

\section{Data}

\subsection{GPS Data}

Since 1992, the CDDIS has supported GNSS data and product archiving for the International GNSS Service (IGS) as one of four global data centers. In this capacity, the CDDIS provides online access to the GNSS data generated by the IGS network as well as the IGS standard, working group, and pilot project products derived from these data.

Hablar de donde salen, el fichero ti me lo ha dado manuel. El fichero ti es el resultado de "pre-procesar" los datos Raw que hay en la web de ftp://cddis.nasa.gov/.















\cite{cddisnasa} -> The Crustal Dynamics Data Information System (CDDIS) was initially developed to provide a central data bank for NASA's Crustal Dynamics Project (CDP). The system continues to support the space geodesy and geodynamics community through NASA's Space Geodesy Project as well as NASA's Earth Science Enterprise. The CDDIS was established in 1982 as a dedicated data bank to archive and distribute space geodesy related data sets. Today, the CDDIS archives and distributes mainly Global Navigation Satellite Systems (GNSS, currently Global Positioning System GPS and GLObal NAvigation Satellite System GLONASS), laser ranging (both to artificial satellites, SLR, and lunar, LLR), Very Long Baseline Interferometry (VLBI), and Doppler Orbitography and Radio-positioning Integrated by Satellite (DORIS) data for an ever increasing user community of geophysists.
The CDDIS is operational on a dedicated computer located at the Goddard Space Flight Center in Greenbelt, MD.
The CDDIS has served as a global data center for the International GNSS Service (IGS) since 1992. The CDDIS also actively supports the International Laser Ranging Service (ILRS), the International VLBI Service for Geodesy and Astrometry (IVS), International DORIS Service (IDS), and the International Earth Rotation and Reference Systems Service (IERS) as a global data center.
To learn more about these space geodetic techniques and their respective CDDIS data holdings, click on the images below.

\cite{noll2010crustal} -> Since 1982, the Crustal Dynamics Data Information System (CDDIS) has supported the archive and distribution of geodetic data products acquired by the National Aeronautics and Space Administration (NASA) as well as national and international programs. The CDDIS provides easy, timely, and reliable access to a variety of data sets, products, and information about these data. These measurements, obtained from a global network of nearly 650 instruments at more than 400 distinct sites, include DORIS (Doppler Orbitography and Radiopositioning Integrated by Satellite), GNSS (Global Navigation Satellite System), SLR and LLR (Satellite and Lunar Laser Ranging), and VLBI (Very Long Baseline Interferometry). The CDDIS data system and its archive have become increasingly important to many national and international science communities, particularly several of the operational services within the International Association of Geodesy (IAG) and its observing system the Global Geodetic Observing System (GGOS), including the International DORIS Service (IDS), the International GNSS Service (IGS), the International Laser Ranging Service (ILRS), the International VLBI Service for Geodesy and Astrometry (IVS), and the International Earth rotation and Reference frame Service (IERS). Investigations resulting from the data and products available through the CDDIS support research in many aspects of Earth system science and global change. Each month, the CDDIS archives more than one million data and derived product files totaling over 90 Gbytes in volume. In turn, the global user community downloads nearly 1.2 Tbytes (over 10.5 million files) of data and products from the CDDIS each month. The requirements of analysts have evolved since the start of the CDDIS; the specialized nature of the system accommodates the enhancements required to support diverse data sets and user needs. This paper discusses the CDDIS, including background information about the system and its user communities, archive contents, available metadata, and future plans.


Explicar que es cddis, como se procesan los datos, link al \cite{hernandez2009igs}, ahi esta explicado como los procesa

For this section, only these pre-processed Ti files were needed, as detecting the flares in real time is a task that will be discussed later in the project.

\subsection{The Halloween Storm}

To test this, the data used was that of the called "Hallowen Storm, in which 

Halloween storm, poner ejemplos, citar al texto, porque esta tormenta otros papers de la storm tambien, decir que usaremos el momento exacto 11.05

\subsection{Formatting}
hablar de los datos, el formato, relevant to this algorithm: mappingion, d2li, etc

\subsection{AWK}

Usaremos awk, como va, muy breve




With this data we can obtain the two parameters that will yield the previously shown plot: the VTEC value and the cosine of the solar-zenith angle

\section{VTEC}

Fisrt we wanted to obtain VTEC distribution throughout the day, to visually see if any spikes appeared confirming that the moment we were going to study based on the paper \cite{hernandez2012gnss} was correct.

For each epoch in our data set (from 10.5 to 11.5 with a sampling rate of 30 seconds) we needed to compute de VTEC value.

\subsection{Computing the VTEC}

The VTEC can be estimated using the following approach:

bla bla bla justify the following equation

\begin{equation} \label{eq:1}
	\frac{d^{2}V}{dt^{2}} = \frac{d^{2}Li}{M}
\end{equation}

Where $M=\frac{1}{\cos Z}$ is the "ionospheric mapping function", given by the inverted cosine of the satellite-zenith angle that we have for each IPP. \cite{hernandez2012gnss}

Therefore, we can estimate the VTEC of an IPP by dividing two of the given fields in the data file:
\begin{equation} \label{eq:2}
	VTEC \approx \frac{d^{2}Li}{M}
\end{equation}
The Vtec...formulas...d2li/mappingio


\subsection{Distribution throughout the day}

Because the only operation that had to be performed was the previous division, a simple AWK script was used to filter out the two necessary fields from the data file (d2li and  mapping) and print the resulting value and the time. 

\begin{lstlisting}[language=Awk, caption=process]
{
	/a/
	d2li = $21;
	mappingFunc = $43;
	vtec = d2li/mappingFunc;
	print $3 " " vtec
}
\end{lstlisting}

\begin{lstlisting}[language=Bash, caption=Bash script to execute the procedures]
#!/bin/bash
tiDataFile="../data/ti.2003.301.10h30m-11h30m.gz"

zcat "$tiDataFile" | gawk -f previewVTECDistribution.awk > vtecValues
gnuplot -e "set terminal png; set output 'vtecDistribution.png'; set title 'VTEC Distribution'; set xlabel 'Time of the day (hours)'; set ylabel 'VTEC'; set grid; plot \"vtecValues\" using 1:2 with point"
rm vtecValues
\end{lstlisting}
\clearpage

The bash script executes the AWK process with the data as the input and outputs n rows with two columns: the time of the day and the calculated VTEC, plotting the results using Gnuplot.
The results can be seen in figure \ref{fig:vtecDistribution}, where we can see how the VTEC value evolves throught the day. Visually, a spike can be seen between 11 and 11.2 hours.

\begin{figure}[!htb]
	\begin{subfigure}[b]{0.5\textwidth}
		\includegraphics[width=\linewidth]{images/ch4/vtecDistributionGeneral.png}
		\caption{All IPPs}
	\end{subfigure}
	\hfill
	\begin{subfigure}[b]{0.5\textwidth}
		\includegraphics[width=\linewidth]{images/ch4/vtecDistributionVill.png}
		\caption{Villafranca station}
	\end{subfigure}
	\caption{VTEC distribution throughout the day for all IPPS (a) and for IPPs that have Vill as the receiver (b)}
	\label{fig:vtecDistribution}
\end{figure}

To see the event more clearly, though, we can focus on one specific receiver (which will still yield multiple IPPs, as the receiver works with different satellites). For the particular case of the Villafranca, Spain station (identified as Vill), we obtain the plot from figure \ref{fig:vtecDistribution:b}. At this time of day around 11:00h the Sun would have a greater effect on the IPPs of this station, so the spike can be seen more clearly. 

As mentioned before the flare took place at 11.05, so we could proceed using this data range and this epoch in particular.

\section{Solar-zenith angle}

The solar-zenith angle (which will be denoted \chi from now onwards) plays a major role when studying this event: it is the angle formed by the Sun and the Earth's zenith and indicates the effect the flare is having on a particular IPP. It is expected that this variable presents a correlation with the increase in VTEC, which is what we aim to observe in this chapter.

Figure \ref{fig:solar-zenith-angle}, at the end of the chapter, provides a visual representation of this variable that along with the results depicts how it can affect the VTEC value. 

Obtaining the angle between two celestial objects has been shown in the previous section by means of equations \ref{eq:1}, \ref{eq:2} and \ref{eq:3}, when calculating the angle between the Sun and a detected GRB.

\begin{equation} \label{eq:1}
unitVectorObjectA =	
\begin{bmatrix}
\cos\delta_{g} * \cos\alpha_{g} \\ 
\cos\delta_{g} * \sin\alpha_{g} \\
\sin\delta_{g}
\end{bmatrix}
\end{equation}

\begin{unitVectorObjectB} \label{eq:2}
unitVectorSun =	
\begin{bmatrix}
\cos\delta_{s} * \cos\alpha_{s} \\ 
\cos\delta_{s} * \sin\alpha_{s} \\
\sin\delta_{s}
\end{bmatrix}
\end{equation}

\begin{equation} \label{eq:3}
\cos \chi = unitVectorObjectA \cdot unitVectorObjectB
\end{equation}\\

For this case, though, the angle is computed using the IPP's Right Ascension and Latitude (equivalent to declination), yielding the cosine of $\chi$, the \textbf{solar-zenith angle}.

USING THE OTHER EQUIATION explain

The previous computation is done for every IPP (every line outputed by the AWK script) as shown in the following Fortran code:

\begin{lstlisting}[language=Fortran, caption=Reading the content of the file and storing the computations for plotting
subroutine traverseFile (raSun, decSun)
implicit none
	real :: raIPP, decIPP, mapIon, d2Li, cosX, vtec
	real, intent(in) :: raSun, decSun

	350 format  (F10.4, F10.4, F15.10, F15.10)
	
	do while (1 == 1)
		read (1, *, end = 240) raIPP, decIPP, mapIon, d2Li
		raIPP = toRadian(raIPP)
		decIPP = toRadian(decIPP)
		cosX = computeSolarZenithAngle(raIPP, decIPP, raSun, decSun)
		vtec = estimateVTEC(mapIon, d2Li)
		if (vtec < 20 .and. vtec > -0.4) then
			write (*, 350) cosX, vtec
		end if
	end do
	240 continue
end subroutine traverseFile
\end{lstlisting}

\begin{lstlisting}[language=Fortran, caption=Computation of the solar-zenith a angle
real function computeSolarZenithAngle (raIPP, decIPP, raSun, decSun)
			implicit none

			real, intent(in) :: raIPP, decIPP, raSun, decSun
			real :: solarZenithAngle
			
			solarZenithAngle = sin(decIPP)*sin(decSun) + cos(decIPP)*cos(decSun)*cos(raIPP - raSun)
			return
		end function computeSolarZenithAngle
\end{lstlisting}

\section{Results}

Taking $212.338º$ and $-13.060º$ as the Sun's right ascension and declination, respectively, and the measurements of all IPPs at 11.05 hours, figure \ref{comparison} shows both the result obtained in the paper "GNSS measurement of EUV photons flux rate during strong and mid solar flares" \ref{hernandez2012gnss} (left) and the result of plotting the output of our program (right).

\begin{figure}[!htb]
	\begin{subfigure}[b]{0.5\textwidth}
		\includegraphics[width=\linewidth]{images/ch4/vtecDistributionGeneral.png}
		\caption{Result from GNSS measurement of EUV photons flux rate during strong and mid solar flares}
	\end{subfigure}
	\hfill
	\begin{subfigure}[b]{0.5\textwidth}
		\includegraphics[width=\linewidth]{images/ch4/vtecDistributionVill.png}
		\caption{Our result}
	\end{subfigure}
	\caption{VTEC distribution throughout the day for all IPPS (a) and for IPPs that have Vill as the receiver (b)}
	\label{fig:comparison}
\end{figure}

As we can observe, there is a strong relation between the cosine of the solar-zenith angle and theVTEC content, which increases from \cos\chi = 0 (90º) to \cos\chi = 1 (0º) and it doesn't seem to affect it from \cos\chi = -1 (180º) to \cos\chi = 1 (0º).



In conclusion, we can see that there appears to be correlation between the two variables. This correlation will be studied more in detail in the following section, where a first approach of the algorithm will be presented to detect the flare without knowing its location.













