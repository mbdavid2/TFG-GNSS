\documentclass[12pt]{article}

%%% Packages %%%

\usepackage[utf8]{inputenc}
\usepackage{titlesec}
\usepackage{graphicx}
\usepackage[margin=1in]{geometry}
\usepackage{eurosym}
\usepackage{makecell}

%%% Styles %%%

\titleformat{\chapter}{\normalfont\huge}{\thechapter.}{20pt}{\huge\it}

\titleformat{\paragraph}
{\normalfont\normalsize\bfseries}{\theparagraph}{1em}{}
\titlespacing*{\paragraph}
{0pt}{3.25ex plus 1ex minus .2ex}{1.5ex plus .2ex}

\bibliographystyle{abbrv}

%%% Document information %%%

\title{
	{
	\includegraphics[width=0.7\linewidth]{logo-fib.png}	
	\vspace{1cm}
	\textbf{\\Budget and sustainability} \\
	\large New real-time GNSS algorithms for detection and measurement of potential geoeffective stellar flares}
\author{\textbf{Author}\\
	David Moreno Borr\`as
	\\ \\
	\textbf{Supervisor}\\
	 Manuel Hernández-Pajares}
\date{\today}
}


%%%%%%%%%%%%%%%%%%%%%%%%%%%%%%%%%%%
%%%%%%% Start of document %%%%%%%%%
%%%%%%%%%%%%%%%%%%%%%%%%%%%%%%%%%%%

\begin{document}
	
\pagenumbering{arabic}
\clearpage
\maketitle
\thispagestyle{empty}
\clearpage


\tableofcontents
\thispagestyle{empty}
\clearpage

%%%%%%%%%%%%%%%%%%%%%%%%%%%%%%
%%%%%%% Cost estimation %%%%%%
%%%%%%%%%%%%%%%%%%%%%%%%%%%%%%

\section{Cost estimation}

In the following sections, an estimation of the cost is presented. These are going to be divided in four major sections: hardware, software, human resources and indirect costs. Because  some of the tasks will use the same resources, they have been grouped, but those that use different resources will be studied in a different section.

\subsection{Software resources}

\paragraph{Common resources}

\begin{table}[h!]
	\centering
	\def\arraystretch{1.2}
	\begin{tabular}{|c c c c c|} 
		\hline
		Product & Units & Price & Useful life (years) & Amortization \\ [0.5ex] 
		\hline\hline
		Ubuntu 18.04 & 1 & 0 \euro & - & 0 \euro \\ 
		\hline
		Google Chrome & 1 & 0 \euro & - & 0 \euro \\
		\hline
		Evince & 1 & 0 \euro & - & 0 \euro \\
		\hline\hline
		Total &  & 0 \euro &  & 0 \euro \\
		\hline
	\end{tabular}
	\caption{Software costs}
\end{table}

\paragraph{Developing the algorithms}

\begin{table}[h!]
	\centering
	\def\arraystretch{1.2}
	\begin{tabular}{|c c c c c|} 
		\hline
		Product & Units & Price & Useful life (years) & Amortization \\
		\hline\hline
		Git & 1 & 0 \euro & - & 0 \euro \\ 
		\hline
		GitHub & 1 & 0 \euro & - & 0 \euro \\
		\hline
		Sublime Text 3 & 1 & 0 \euro & - & 0 \euro \\
		\hline
		Python & 1 & 0 \euro & - & 0 \euro \\
		\hline
		GNSS Data & 1 & 0 \euro & - & 0 \euro \\
		\hline
		GFortran & 1 & 0 \euro & - & 0 \euro \\
		\hline\hline
		Total &   & 0 \euro &  & 0 \euro \\
		\hline
	\end{tabular}
	\caption{Software costs}
\end{table}

\clearpage

\paragraph{GEP and writing the report}

\begin{table}[h!]
	\centering
	\def\arraystretch{1.2}
	\begin{tabular}{|c c c c c|} 
		\hline
		Product & Units & Price & Useful life (years) & Amortization \\
		\hline\hline
		LibreOffice & 1 & 0 \euro & - & 0 \euro \\ 
		\hline
		LaTeX & 1 & 0 \euro & - & 0 \euro \\
		\hline
		TeamGantt & 1 & 0 \euro & - & 0 \euro \\
		\hline\hline
		Total &   & 0 \euro &  & 0 \euro \\
		\hline
	\end{tabular}
	\caption{Software costs}
\end{table}

\subsection{Hardware resources}

The following table contains the costs of the hardware that is going to be used for the project. These resources are common to all phases.

\begin{table}[h!]
	\centering
	\def\arraystretch{1.2}
	\begin{tabular}{|c c c c c|} 
		\hline
		Product & Units & Price & Useful life (years) & Amortization \\
		\hline\hline
		Asus X555L & 1 & 750 \euro & 6 & 60 \euro \\ 
		\hline
		PC devices & 1 & 200 \euro & 6 & 20 \euro \\
		\hline\hline
		Total &   & 950 \euro &  & 80 \euro \\
		\hline
	\end{tabular}
	\caption{Hardware costs}
\end{table}

\subsection{Human resources}

The project is going to be developed by one person, which will have to be the project manager, software developer and tester. 

We estimated in the planning section a total dedication time for the project of 550 hours, so here we present an estimation of the distribution of those hours between the roles and the cost of each.

\begin{table}[h!]
	\centering
	\def\arraystretch{1.2}
	\begin{tabular}{|c c c c |} 
		\hline
		Role & \euro/hour & Hours & Cost \\
		\hline\hline
		Project manager & 45 & 100 & 4500 \\
		\hline
		Software developer & 40 & 300 & 12000 \\
		\hline
		Tester & 30 & 150 & 4500 \\
		\hline\hline
		Total &  & 550 & 21000 \\
		\hline
	\end{tabular}
	\caption{Human resources costs}
\end{table}

\subsection{Indirect costs}

Indirect costs of elements that will be needed in order to use the previous hardware are shown in table 6:

We can estimate the energy expenditure during the project assuming the computer consumes an average of 200 watts per hour. If we plan to use it during the 550 hours of the project, we cam estimate a total of 110 kW spent.

\begin{table}[h!]
	\centering
	\def\arraystretch{1.2}
	\begin{tabular}{|c c c c|} 
		\hline
		Product & Use & Price & Estimated cost \\
		\hline\hline
		ADSL & 4 months & 40 \euro/month & 160 \euro\\
		\hline
		Electricity & 110 kWh & 0.1067 \euro/kWh & 11.7 \euro\\
		\hline\hline
		Total &  &  & 172 \euro\\
		\hline
	\end{tabular}
	\caption{Indirect costs}
\end{table}

\subsection{Total budget}

In the following table, the total cost of the project can be seen, estimated using data seen in the previous tables.

As we can see, there is no software cost because only open-source or free tools have been used.

\begin{table}[h!]
	\centering
	\def\arraystretch{1.2}
	\begin{tabular}{|c c|} 
		\hline
		Resource & Estimated cost\\
		\hline\hline
		Software & 0 \euro\\
		\hline
		Hardware & 950 \euro\\
		\hline
		Human resources & 21000 \euro\\
		\hline
		Indirect costs & 172 \euro\\
		\hline\hline
		Total & 22122 \euro\\
		\hline
	\end{tabular}
	\caption{Total cost of the project}
\end{table}

\subsection{Budget control}

As seen in the planning section, some of the tasks may take longer than estimated because of unexpected difficulties, which would in turn increase the total cost of the project. So we have to consider the fact that these delays could lead to an increase in the total cost of the project.

Although unlikely, hardware faults might occur that would require more resources, but the main factor that might influence the budget during the project is time, which would increase the amount of work hours done by either the project manager, the software developer or the tester.

\section{Sustainability}

In this section we will focus on evaluating the impact of our project by studying its sustainability in three different aspects: environmental, economical and social.

The analysis is going to be based on the application of the following sustainability matrix:

\begin{table}[h!]
\begin{center}
	\def\arraystretch{2.2}
	\begin{tabular}{|c|c|c|c|}
		\hline
		\thead{} & \thead{\textbf{PPP}} & \thead{\textbf{Exploitation}} & \thead{\textbf{Risks}} \\
		\hline
		\textbf{Environmental} & \makecell{Low design \\ consumption} & \makecell{Low ecological\\ footprint} & \makecell{Low environmental\\ risks}\\
		\hline
		\textbf{Economic} & \makecell{Medium \\ resources needed} & Low cost & \makecell{High cost, more\\human resources}\\
		\hline
		\textbf{Social} & \makecell{High personal \\ impact} & \makecell{Medium social  \\ impact} & \makecell{Low social risks}\\
		\hline
	\end{tabular}
\caption{Sustainability matrix}
\end{center}
\end{table}

\subsection{Environmental sustainability}

During the project we are going to use the minimum amount of resources possible, which have been presented in the Cost estimation section. Because what we are going to use is mainly software, the resource from the project which will have an environmental impact is going to be the energy spent by the devices running during the project (the computer).

Furthermore, if the project is successful, it would present an alternative to currently working satellites that detect Gamma-Ray Bursts (GRB), like the Gamma-ray Large Area Space Telescope (GLAST) or weather satellites like the Geostationary Operational Environmental Satellite (GOES).

As seen in its specification manual (https://www.nasa.gov/pdf/221503main\_GLAST-041508.pdf) GLAST needs about 1500 watts average over an orbit, which is significantly more than the consumption of the computer that we have estimated before: 110 kWh. The telescope, however, is equipped with solar panels that can supply up to 3122 watts in sunlight. 

While our alternative would not obtain results with the precision and information that these missions aim to achieve, some results would be similar, so we can also consider the difference in environmental impact between both.

The larger environmental impact of the GLAST mission, however, lies in the design, build and launch of the telescope. While information about the cost of the previous factors is available and will be studied in the next section, there is no information provided regarding its environmental impact, although we can say that it likely has a considerably larger one than that of our project, in which only a computer is used.

In conclusion, the project’s resources are mainly software and the factor with the biggest environmental impact will be the energy expenditure of the computer, which is significantly lower than that of the currently existing alternatives.

\subsection{Economic sustainability}

In previous sections we have studied the cost of our project (hardware, software and human resources). From an economical point of view, our project presents an alternative to telescopes like GLAST or GOES. Albeit less precise and equipped, some of its aspects and purposes are shared.

The  cost to design, build and launch GLAST, for example, had a total international contribution of 690 US dollars. Considering our project relies only on free or open-source data and software, it would be offering an alternative with a lower economical impact. 

It would be difficult to do this project with a lower cost, considering the only resource that has an economical impact besides the human work is the hardware (a computer). It would be difficult to lower the costs of this area considering a computer is needed for most computer science projects.

\subsection{Social sustainability}

Personally, the project is very relevant to me. I wanted to work on a project to see how computer science could be applied to a field like astronomy or physics. I think the project and algorithms we are developing are a good example of the place CS has in this fields and the role it plays.

It has also helped me gaining experience in terms of information retrieval and research. Both writing reports like this one, planning projects and researching information from reputable sources that can be used in our project.

If the project is successful, it could turn into a useful tool for astronomers that could be used as an astronomical instrument to measure the Sun’s EUV using only open-source GPS data, rather than a dedicated telescope, which would be a useful, less expensive alternative.









\end{document}

